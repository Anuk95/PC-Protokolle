\documentclass[12pt,a4paper,titlepage,headinclude,bibtotoc]{scrartcl}

%---- Allgemeine Layout Einstellungen ------------------------------------------

% Für Kopf und Fußzeilen, siehe auch KOMA-Skript Doku
\usepackage[komastyle]{scrpage2}
\pagestyle{empty}
\setheadsepline{0.5pt}[\color{black}]
\automark[section]{chapter}


%Einstellungen für Figuren- und Tabellenbeschriftungen
\setkomafont{captionlabel}{\sffamily\bfseries}
\setcapindent{0em}


%---- Weitere Pakete -----------------------------------------------------------
% Die Pakete sind alle in der TeX Live Distribution enthalten. Wichtige Adressen
% www.ctan.org, www.dante.de

% Sprachunterstützung
\usepackage[ngerman]{babel}

% Benutzung von Umlauten direkt im Text
% entweder "latin1" oder "utf8"
\usepackage[utf8]{inputenc}

% Pakete mit Mathesymbolen und zur Beseitigung von Schwächen der Mathe-Umgebung
\usepackage{latexsym,exscale,stmaryrd,amssymb,amsmath}


\usepackage[nointegrals]{wasysym}
\usepackage{eurosym}

% Anderes Literaturverzeichnisformat
%\usepackage[square,sort&compress]{natbib}
\usepackage{hyperref}
% Für Farbe
\usepackage{color}
\usepackage{graphicx}
\usepackage{wrapfig}
\usepackage{subfigure}

% Caption neben Abbildung
\usepackage{sidecap}

% Befehl für "Entspricht"-Zeichen
\newcommand{\corresponds}{\ensuremath{\mathrel{\widehat{=}}}}
% Befehl für Errorfunction
\newcommand{\erf}[1]{\text{ erf}\ensuremath{\left( #1 \right)}}

%Fußnoten zwingend auf diese Seite setzen
\interfootnotelinepenalty=1000

%Für chemische Formeln (von www.dante.de)
%% Anpassung an LaTeX(2e) von Bernd Raichle
\makeatletter
\DeclareRobustCommand{\chemical}[1]{%
  {\(\m@th
   \edef\resetfontdimens{\noexpand\)%
       \fontdimen16\textfont2=\the\fontdimen16\textfont2
       \fontdimen17\textfont2=\the\fontdimen17\textfont2\relax}%
   \fontdimen16\textfont2=2.7pt \fontdimen17\textfont2=2.7pt
   \mathrm{#1}%
   \resetfontdimens}}
\makeatother

%Honecker-Kasten mit $$\shadowbox{$xxxx$}$$
\usepackage{fancybox}

%SI-Package
\usepackage{siunitx}

%keine Einrückung, wenn Latex doppelte Leerzeile
\parindent0pt

%Bibliography \bibliography{literatur} und \cite{gerthsen}
%\usepackage{cite}
\usepackage{babelbib}
\selectbiblanguage{ngerman}

\begin{document}

\begin{titlepage}
\centering
\textsc{\Large Praktikum zur Einführung in die physikalische Chemie,\\[1.5ex] Universität Göttingen}

\vspace*{3cm}

\rule{\textwidth}{1pt}\\[0.5cm]
{\huge \bfseries
  Interferenz\\[1.5ex]
  und Wellenlängenmessung}\\[0.5cm]
\rule{\textwidth}{1pt}

\vspace*{3cm}

\begin{Large}
\begin{tabular}{ll}
Praktikant: &  Alea Tokita\\
Versuchspartner: &  Julia Stachowiak\\
 E-Mail: & alea.tokita@stud.uni-goettingen.de\\
 Betreuer: & ??\\
 Versuchsdatum: & 09.11.2015\\
\end{tabular}
\end{Large}

\vspace*{0.8cm}

\begin{Large}
\fbox{
  \begin{minipage}[t][2.5cm][t]{6cm} 
    Eingegangen am:
  \end{minipage}
}
\end{Large}

\end{titlepage}

\newpage


\textbf{Interferenz und Wellenlängenmessung am 9.11.2015} \\

Name: Julia Stachowiak \\
Versuchspartner: Alea Tokita \\
Assistent: \\ \\



\begin{table}
\centering
\begin{large}

\end{large}
\begin{tabular}{|p{4 cm}||p{4 cm}|p{4 cm}|}
        \hline
          Abstand  & Gitter 1  & Gitter 2 \\
          \textit{in cm} & & \\
         
         
         \hline 
         $y_3 $& & \\
         \hline
         $y_2 $& & \\
         \hline
         $y_{1} $& & \\
         
         \hline
         $y_{-1}$& & \\
         \hline
         $y_{-2}$& & \\
         \hline             
         $y_{-3}$& & \\
         \hline
\end{tabular}
\end{table}


\section{Aufbau}
Alles war total super!!!!!


\section{Auswertung}

Ja aber am Schluss ist es explodiert!

\section{Fehlerrechnung}
Wir machen keine Fehler.


\section{Fehlerdiskussion}
Das muss man nicht mehr diskutieren.

\subsection{Diskussion systematischer Fehler}



\subsection{Vergleich mit Literaturwerten}




\end{document}


