\documentclass[11pt,a4paper,titlepage,headinclude,bibtotoc]{scrartcl}

%---- Allgemeine Layout Einstellungen ------------------------------------------

% Für Kopf und Fußzeilen, siehe auch KOMA-Skript Doku
\usepackage[komastyle]{scrpage2}
\pagestyle{plain}
\setheadsepline{0.5pt}[\color{black}]
\automark[section]{chapter}


%Einstellungen für Figuren- und Tabellenbeschriftungen
\setkomafont{captionlabel}{\sffamily\bfseries}
\setcapindent{0em}


%---- Weitere Pakete -----------------------------------------------------------
% Die Pakete sind alle in der TeX Live Distribution enthalten. Wichtige Adressen
% www.ctan.org, www.dante.de

% Sprachunterstützung
\usepackage[ngerman]{babel}

% Benutzung von Umlauten direkt im Text
% entweder "latin1" oder "utf8"
\usepackage[utf8]{inputenc}

% Pakete mit Mathesymbolen und zur Beseitigung von Schwächen der Mathe-Umgebung
\usepackage{latexsym,exscale,stmaryrd,amssymb,amsmath}


\usepackage[nointegrals]{wasysym}
\usepackage{eurosym}

% Anderes Literaturverzeichnisformat
%\usepackage[square,sort&compress]{natbib}
\usepackage{hyperref}
% Für Farbe
\usepackage{color}
\usepackage{graphicx}
\usepackage{wrapfig}
\usepackage{subfigure}

% Caption neben Abbildung
\usepackage{sidecap}

% Befehl für "Entspricht"-Zeichen
\newcommand{\corresponds}{\ensuremath{\mathrel{\widehat{=}}}}
% Befehl für Errorfunction
\newcommand{\erf}[1]{\text{ erf}\ensuremath{\left( #1 \right)}}

%Fußnoten zwingend auf diese Seite setzen
\interfootnotelinepenalty=1000

%Für chemische Formeln (von www.dante.de)
%% Anpassung an LaTeX(2e) von Bernd Raichle
\makeatletter
\DeclareRobustCommand{\chemical}[1]{%
  {\(\m@th
   \edef\resetfontdimens{\noexpand\)%
       \fontdimen16\textfont2=\the\fontdimen16\textfont2
       \fontdimen17\textfont2=\the\fontdimen17\textfont2\relax}%
   \fontdimen16\textfont2=2.7pt \fontdimen17\textfont2=2.7pt
   \mathrm{#1}%
   \resetfontdimens}}
\makeatother

%Honecker-Kasten mit $$\shadowbox{$xxxx$}$$
\usepackage{fancybox}

%SI-Package
\usepackage{siunitx}

%keine Einrückung, wenn Latex doppelte Leerzeile
\parindent0pt

%Bibliography \bibliography{literatur} und \cite{gerthsen}
%\usepackage{cite}
\usepackage{babelbib}
\selectbiblanguage{ngerman}

\begin{document}

\centering

\section*{V1: Bestimmung der Loschmidt- Zahl: Messwerte}

\begin{flushleft}
\begin{Large}
\begin{tabular}{ll}
Durchführende: &  Alea Tokita, Julia Stachowiak\\
Assistentin: & Annemarie Kehl\\
 Versuchsdatum: & 09.11.2015\\
\end{tabular}
\end{Large}
\end{flushleft}


\subsection*{Aluminiumkristall}

Gitteronstante $a= 4,049\AA$

\begin{table} [h]
\centering
\begin{tabular}{|p{3 cm}||p{3 cm}|p{3 cm}|p{3 cm}|}
        \hline
		Messung n& Masse $m$ in & Durchmesser $d$ in & Länge $l$ in\\
         \hline 
         1& & &\\
          \hline 
         2& & &\\
          \hline 
         3& & &\\
          \hline 
         4& & &\\
          \hline 
         5& & &\\
          \hline 
         6& & &\\
          \hline 
         7& & &\\
          \hline 
         8& & &\\
          \hline 
         9& & &\\
          \hline 
         10& & &\\
         \hline
         Mittelwert $\bar{x}$ & & &\\
         \hline
\end{tabular}
\end{table}

\subsection*{Lithiumfluorid-Kristall}

Gitteronstante $a= 4,018\AA$

\begin{table} [h!]
\centering
\begin{tabular}{|p{3 cm}||p{3 cm}|p{3 cm}|p{3 cm}|}
        \hline
		Messung n& Masse $m$ in & Durchmesser $d$ in & Länge $l$ in\\
         \hline 
         1& & &\\
          \hline 
         2& & &\\
          \hline 
         3& & &\\
          \hline 
         4& & &\\
          \hline 
         5& & &\\
          \hline 
         6& & &\\
          \hline 
         7& & &\\
          \hline 
         8& & &\\
          \hline 
         9& & &\\
          \hline 
         10& & &\\
         \hline
         Mittelwert $\bar{x}$ & & &\\
         \hline
\end{tabular}
\end{table}

\subsection*{Kalziumfluorid-Kristall}

Ionenabstand $r=2,365\AA$

\begin{table} [h!]
\centering
\begin{tabular}{|p{3 cm}||p{3 cm}|p{3 cm}|p{3 cm}|}
        \hline
		Messung n& Masse $m$ in & Durchmesser $d$ in & Länge $l$ in\\
         \hline 
         1& & &\\
          \hline 
         2& & &\\
          \hline 
         3& & &\\
          \hline 
         4& & &\\
          \hline 
         5& & &\\
          \hline 
         6& & &\\
          \hline 
         7& & &\\
          \hline 
         8& & &\\
          \hline 
         9& & &\\
          \hline 
         10& & &\\
         \hline
         Mittelwert $\bar{x}$ & & &\\
         \hline
\end{tabular}
\end{table}


\end{document}
